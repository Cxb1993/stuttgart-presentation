\section[Model]{Model}
\begin{frame}
  \frametitle{SDPD}
   \begin{block}{}
   SDPD is an extension of SPH method for mesoscopic scale by introducing thermal fluctuations in a consistent
way through the fluctuation-dissipation theorem.
  \end{block}
 \begin{itemize}
  \item \alert{SPH for Navier-Stokes Equation}
  \item \alert{Thermal fluctuations} %~\footnotemark
  \item \alert{Mechanical Modeling of the Polymer Molecule}%~\footnotemark[2]
%   \item \alert{Evaporation model}
  \end{itemize}

%   \footnotetext{\tiny \bibentry{Hu2006}}
%   \footnotetext[2]{\tiny \bibentry{Cleary1999}}
\end{frame}
\begin{frame}
  \frametitle{SPH for Navier-Stokes Equation}
Navier-Stokes Equation
\begin{equation}\label{equ:masscon}
 \frac{d\rho}{dt}=-\rho\nabla\cdot\mathbf{v},
\end{equation}
\begin{equation}\label{equ:momecon}
 \frac{\mathbf{dv}}{dt}=\mathbf{g}-\frac{1}{\rho}\mathbf{\nabla}p+\mathbf{F}
\end{equation}

SPH formulation:
  \begin{equation}\label{equ:rho}
 \rho_i=m_i \sum_j W_{ij}
\end{equation}
  \begin{equation}\label{equ:momeevo}
 \frac{d\mathbf{v}_{i}^{(p)}}{dt}=-\frac{1}{m_i}\sum_j\left(\frac{p_i}{d_{i}^{2}}+\frac{p_j}{d_{j}^{2}}\right)\frac{\partial W}{\partial r_{ij}}\mathbf{e}_{ij},
\end{equation}
\begin{equation}\label{equ:acceleration}
 \frac{d\mathbf{v}{_i}^{(v)}}{dt}=-\frac{\eta}{m_i}\sum_j\left(\frac{p^i}{d_{i}^{2}}+\frac{p^j}{d_{j}^{2}}\right)\frac{1}{r_{ij}}\frac{\partial W}{\partial r_{ij}}(\mathbf{e}_{ij}\cdot\mathbf{v}_{ij}\mathbf{e}_{ij}+\mathbf{v}_{ij}),
\end{equation}
\end{frame}

\begin{frame}
  \frametitle{Thermal fluctuations}
% The irreversible part of the particle dynamic~\cite{Hu06} in SPH method is
% \begin{eqnarray}\label{equ:thermal}
% & &\dot{m}_i\vert_{irr}=0 \nonumber \\
% & &\dot{\mathbf{P}}_i\vert_{irr}=\eta\sum_j\left(\frac{1}{d_{i}^{2}}+\frac{1}{d_{j}^{2}}\right)\frac{1}{r_{ij}}\frac{\partial W}{\partial r_{ij}}(\mathbf{e}_{ij}\cdot\mathbf{v}_{ij}\mathbf{e}_{ij}+\mathbf{v}_{ij})
% \end{eqnarray}
% According to the GENERIC formalism~\cite{Grmela1997}, thermal fluctuations can be take into account by postulating the mass and the momentum fluctuations
% of particle $i$
% \begin{eqnarray}\label{equ:thermalb}
% & & d\tilde{m}_i=0 \nonumber \\
% & & d\tilde{\mathbf{P}}_i=\sum_j B_{ij}d\bar{\mathscr{W}}_{ij}\cdot\mathbf{e}_{ij}
% \end{eqnarray}
% where $d\bar{\mathscr{W}}_{ij}$ is the traceless symmetric part of a matrix of independent increments of a Wiener process
% $d\mathscr{W}_{ij}=d\mathscr{W}_{ji}$ i.e., $d\bar{\mathscr{W}}_{ij}=(d\mathscr{W}_{ij}+d\mathscr{W}_{ji}^T)/2-tr[d\mathscr{W}_{ij}]\mathbf{I}/D$.
% \end{frame}
% \begin{frame}
The isothermal deterministic irreversible equations are:
\begin{eqnarray}\label{equ:thermalfur}
  & &\dot{m}_i\vert_{irr}=0  \nonumber \\
& &\dot{\mathbf{P}}_i\vert_{irr}=-\sum_j\frac{B_{ij}^2}{4k_BT}
(\mathbf{e}_{ij}\cdot\mathbf{v}_{ij}\mathbf{e}_{ij}+\mathbf{v}_{ij}),
\end{eqnarray}
where $k_B$ is the Boltzmann constant and $T$ is the system temperature, and $B_{ij}$ is
\begin{equation}\label{equ:b}
 B_{ij}=[-4k_BT\eta\left(\frac{1}{d_{i}^{2}}+\frac{1}{d_{j}^{2}}\right)\frac{1}{r_{ij}}\frac{\partial W}{\partial r_{ij}}]^{1/2}.
\end{equation}
\end{frame}

\begin{frame}
  \frametitle{Mechanical Modeling of the Polymer Molecule}
%   \begin{block}{Rule}
%     If a vapor particle has $T>T_s$ and has a liquid particle as a
%     neighbor we create a new vapor particle
%   \end{block}
%   \begin{itemize}
%   \item mass is from a liquid particle ($m_{liquid} >> m_{vapor}$)
%   \item energy goes to vapor particle
%   \item mechanical momentum is distributed between ``new'' and ``old''
%     vapor particles
%   \end{itemize}
Finitely extensible nonlinear elastic(FENE) springs
\begin{equation}\label{equ:fene}
 \mathbf{F}^{FENE}(\mathbf{r}_{ij})=\frac{K\mathbf{r}_{ij}}{1-(r/R_0)^2}
\end{equation}

FENE-E springs
  \begin{equation}\label{equ:feneE}
 \mathbf{F}^{FENE-E}(\mathbf{r}_{ij})=\frac{K(\mathbf{r}_{ij}-\delta)}{1-[(r-\delta)/R_0]^2}
\end{equation}
where $\delta$ is given minimal distance between neighboring beads.
\end{frame}

%%% Local Variables: 
%%% mode: latex
%%% TeX-master: t
%%% End: 
