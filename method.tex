\section[Model]{Model}
\begin{frame}
  \frametitle{Parts of the model}
  \begin{itemize}
  \item Navier-Stokes equations for velocity
  \item Surface tension forces on the interface~\footnotemark
  \item \alert{Heat conduction~\footnotemark[2]}
  \item \alert{Evaporation model}
  \end{itemize}
  \begin{block}{Assumption}
    Evaporation rate is controlled by heat supply
    $\Longleftrightarrow$ vapor contains as little energy as it can
    without condensation $\Longleftrightarrow$ vapor is at $T=T_s$
  \end{block}

  \footnotetext{\tiny \bibentry{Hu2006}}
  \footnotetext[2]{\tiny \bibentry{Cleary1999}}
\end{frame}

\begin{frame}
  \frametitle{Heat conduction}
  \begin{equation}
    \label{eq:heatConductionTerm}
    \frac{\partial U_i}{\partial t} = \sum_j\frac{4 m_i}{\rho_j \rho_i} \frac{k_j k_i}{k_j + k_i} \alert{T_{ij}} \frac{\partial W}{\partial r_{ij}}
  \end{equation}
  \begin{equation}
    \label{eq:ideal}
    U_i = c_{v,i} T_i
  \end{equation}
  For vapor-vapor and liquid-liquid interaction:
  \begin{equation}
    \label{eq:tij}
    T_{ij} = \frac{2T_i \alert{T_j}}{T_i + \alert{T_j}}
  \end{equation}
  For liquid-vapor
  \begin{equation}
    \label{eq:tab}
    T_{ij} = \frac{2T_i \alert{T_s}}{T_i + \alert{T_s}}
  \end{equation}
  where $T_s$ is a saturation temperature
\end{frame}

\begin{frame}
  \frametitle{Evaporation model: particle creation}
  \begin{block}{Rule}
    If a vapor particle has $T>T_s$ and has a liquid particle as a
    neighbor we create a new vapor particle
  \end{block}
  \begin{itemize}
  \item mass is from a liquid particle ($m_{liquid} >> m_{vapor}$)
  \item energy goes to vapor particle
  \item mechanical momentum is distributed between ``new'' and ``old''
    vapor particles
  \end{itemize}
\end{frame}

%%% Local Variables: 
%%% mode: latex
%%% TeX-master: t
%%% End: 
